% !TEX encoding = UTF-8 Unicode

Auch in der Mathematik ist die Sprache die Grundlage von allem. Die Sprache der Mathematik besteht aus \imp{Aussagen}. Aussagen sind Sätze, denen man das Prädikat \imp{wahr(w)} oder \imp{falsch(f)} zuordnen kann. Das nennt man den \imp{Wahrheitsgehalt} der Aussage. \\
\\
Beachte: Sätze oder Alltagssprache sind oft keine Aussagen (\anf{Wie ist das Wetter heute?}) \\
\\
Oft wird von \imp{Axiomen} (Grundaussagen) ausgegangen. Aus diesen kann man nach bestimmten Regeln neue Aussagen bilden. Um diese Regeln einzuführen, verwenden wir \imp{Definitionen} (Vereinbarungen).

\begin{defi}
	Seien $A$ und $B$ Aussagen. Dann sind folgende Sätze Aussagen:
	\begin{itemize}
		\item[a)] $\neg A$  \anf{nicht $A$} (die Negation von A)
		\item[b)] $A\wedge B$  \anf{$A$ und $B$}
		\item[c)] $A\vee B$  \anf{$A$ oder $B$} (einschließendes oder)
	\end{itemize}
\end{defi} 
Der Wahrheitsgehalt dieser Aussagen ist durch Wahrheitstabellen beschrieben.\\

\begin{center}
	\begin{tabular}{|c|c|}
		\hline
		$A$ & $\neg A$ \\
		\hline
		w & f \\
		f & w \\
		\hline
	\end{tabular}
	\hspace{2cm}
	\begin{tabular}{|c|c|c|c|}
		\hline
		$A$ & $B$ & $A\wedge B$ & $A\vee B$ \\
		\hline
		w & w & w & w \\
		w & f & f & w \\
		f & w & f & w \\
		f & f & f & f \\
		\hline
	\end{tabular}
\end{center}

\begin{bsp}
	\begin{itemize}
		\item[]
		\item[] 3 ist eine Primzahl (w)
		\item[] 3 ist keine Primzahl (f)
	\end{itemize}
	\textbf{Vorsicht}: $B$: alle Primzahlen sind ungerade (f) \par
	$\neg B$: nicht alle Primzahlen sind ungerade (w) oder: wenigstens eine Primzahl ist gerade (w)\par
	falsch ist: $\neg B$: keine Primzahl ist ungerade (f)
\end{bsp}

\begin{defi}
	Sind $A$ und $B$ Aussagen, so auch folgende Sätze:
	\begin{itemize}
		\item[d)] $A\Rightarrow B$: \anf{$A$ impliziert $B$} oder \anf{aus $A$ folge $B$} oder \anf{wenn $A$ gilt, dann auch $B$}
		\item[e)] $A\Leftrightarrow B$: \anf{$A$ ist äquivalent zu $B$} oder \anf{$A$ gilt genau dann, wenn $B$ gilt}
	\end{itemize}
\end{defi} 
Die zugehörige Wertetabellen:
\begin{center}
	\begin{tabular}{|c|c|c|c|}
		\hline
		$A$ & $B$ & $A\Rightarrow B$ & $A\Leftrightarrow B$ \\
		\hline
		w & w & w & w \\
		w & f & f & f \\
		f & w & w & f \\
		f & f & w & w \\
		\hline
	\end{tabular}
\end{center}

\textbf{Merke}: 
\begin{itemize}
	\item Aus einer falschen Aussage folgt alles.
	\item \anf{Man kann Implikationen und Äquivalenzen mit Wahrheitstafeln nachprüfen} (, im Sinn der folgenden Proposition..)
\end{itemize}

\begin{prop}
	Für Aussagen $A,B,C$ gelten:
	\begin{itemize}
		\item[i)] $(A\wedge B) \Leftrightarrow (B\wedge A)$ ; $(A\vee B) \Leftrightarrow (B\vee A)$, d.h. \anf{und} und \anf{oder} sind \imp{kommutativ}.
		\item[ii)] $(A\wedge B)\wedge C \Leftrightarrow A\wedge (B\wedge C)$; $(A\vee B)\vee C\Leftrightarrow A\vee (B\vee C)$, d.h. \anf{und} und \anf{oder} sind \imp{assoziativ}.
		\item[iii)] $A\wedge(B\vee C) \Leftrightarrow (A\wedge B)\vee(A\wedge C)$; $A\vee(B\wedge C) \Leftrightarrow (A\vee B)\wedge(A\vee C)$ (\imp{Distributivität}
		\item[iv)] $\neg(\neg$A$)\Leftrightarrow A$
		\item[v)] $\neg(A\vee B)\Leftrightarrow (\neg A)\wedge (\neg B)$; $\neg(A\wedge B) \Leftrightarrow (\neg A)\vee (\neg B)$ (\imp{deMorgansche Regel})
	\end{itemize}
\end{prop}

\begin{bew}[zum Teil]
	\textbf{i)} 1. Teil
	\begin{center}
		\begin{tabular}{|c|c|c|c|}
			\hline
			$A$ & $B$ & $A\wedge B$ & $B\wedge A$ \\
			\hline
			w & W & w & w \\
			w & f & f & f \\
			f & w & f & f \\
			f & f & f & f \\
			\hline
		\end{tabular}
	\end{center}
	\textbf{v)} 1. Teil
	\begin{center}
		\begin{tabular}{|c|c|c|c|c|c|c|}
			\hline
			$A$ & $B$ & $A\vee B$ & $\neg(A\vee B)$ & $\neg A$ & $\neg B$ & $(\neg A)\wedge(\neg B)$ \\
			\hline
			w & W & W & f & f & f & f \\
			w & f & w & f & f & w & f \\
			f & w & w & f & w & f & f \\
			f & f & f & w & w & w & w\\
			\hline
		\end{tabular}
	\end{center}
	Alles Übrige mit Wahrheitstafeln. \hfill $\Box$\
\end{bew}

\begin{prop}
	Für Aussagen $A$ und $B$ gelten:
	\begin{itemize}
		\item[i)] $(A\Rightarrow B)\Leftrightarrow (\neg A\vee B)$
		\item[ii)] $(A\Rightarrow B) \Leftrightarrow (\neg B \Rightarrow \neg A)$ (\imp{Kontraposition})
		\item[iii)] $\neg(A\Rightarrow B) \Leftrightarrow A\wedge\neg B$ (\imp{Widerspruchsbeweis})
	\end{itemize}
\end{prop}

\begin{inter}
	\begin{itemize}
		\item[]
		\item[ii)] Um zu zeigen, dass $B$ aus $A$ folgt, kann man alternativ zeigen, dass aus $\neg B$ die Aussage $\neg A$ folgt.
		\item[iii)] Um $A\Rightarrow B$ zu zeigen, kann man wie folgt vorgehen: $A$ gelte und man nimmt an, dass $B$ falsch ist und dann folgt $\neg(A\Rightarrow B)$ ist falsch, dann folgt $A\Rightarrow B)$ gilt. (Widerspruchsbeweis)
	\end{itemize}
\end{inter}

\begin{prop}
	Für Aussagen $A,B$ und $C$ gelten:
	\begin{itemize}
		\item[i)] $(A\Rightarrow B)\wedge(B\Rightarrow C)\Rightarrow(A\Rightarrow C)$
		\item[ii)] $(A\Leftrightarrow B)\Leftrightarrow((A\Rightarrow B)\wedge(B\Rightarrow A))$
	\end{itemize}
\end{prop}

\begin{bew}
	Wahrheitstafeln,
\end{bew}

\begin{inter}
	ii) sagt: gehe in 2 Schritten vor, um $\Leftrightarrow$ nachzuweisen!
\end{inter}

\begin{bew}[von 0.4ii)]
	$(A\Rightarrow B) \Leftrightarrow \neg A\vee B \Leftrightarrow B\vee\neg A \Leftrightarrow \neg(\neg B)\vee(\neg A) \Leftrightarrow (\neg B \Rightarrow \neg A)$\hfill $\Box$
\end{bew}